\section{Introducción}


El scheduling es un método mediante el cual se le da a las tareas de un computador acceso a lo recursos del mismo. Se busca que un scheduler cumpla diversos requisitos, como ser justo en cuanto a los recursos que se le asigna a cada proceso, conseguir terminar una gran cantidad de procesos en poco tiempo, etc.
En este trabajo se estudiaran distintos schedulers sobre distintos lotes de tareas y el efecto que distintas características  tienen sobre el desempeño de cada uno, por ejemplo quantum, cantidad de cores, tiempo de migración, de cambio de contexto,  cantidad de tareas etc. Para esto se utilizará un simulador que dadas las implementaciones de estos schedulers nos permite obtener información sobre como se comportan con un lote de tareas dado. 